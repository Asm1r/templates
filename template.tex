% Template asmir [at] archlinux [dot] us

\documentclass[12pt,a4paper]{report}

\usepackage[DIV=15]{typearea}
\usepackage[utf8]{inputenc}
\usepackage{mathtools}
\usepackage{amsmath}
\usepackage{multicol}
\usepackage[usenames,dvipsnames]{xcolor}
\usepackage{fancyhdr}
\usepackage{caption}
\usepackage{enumerate}
\usepackage{float}
\usepackage{listings}
\usepackage{tikz}
\usepackage{gnuplot-lua-tikz}
\usepackage{embedfile}

%% Add source files to final document
\embedfile{template.tex}

% For faster processing, load Matlab syntax for listings
\lstloadlanguages{Octave}%
\lstset{language=Octave,
	frame=single,
	basicstyle=\small\ttfamily,
	keywordstyle=[1]\color{Orchid}\bf,
	keywordstyle=[2]\color{Brown},
	keywordstyle=[3]\color{Blue},
	identifierstyle=\color{BlueViolet},
	%commentstyle=\usefont{T1}{pcr}{m}{sl}\color{OliveGreen}\small,
	stringstyle=\color{MidnightBlue},
	showstringspaces=false,
	tabsize=8,
	%%% Put standard Octave functions not included in the default
	%%% language here
	morekeywords={},
	%%% Put Octave function parameters here
	morekeywords=[2]{},
	%%% Put user defined functions here
	morekeywords=[3]{},
	numbers=left,                           % Line numbers on left
	firstnumber=1,                          % Line numbers start with line 1
	numberstyle=\tiny\color{Blue},          % Line numbers are blue
	stepnumber=5                            % Line numbers go in steps of 5
	}

%% Header and fotter
\pagestyle{fancy}
\chead{\textsc{Header}}

\begin{document}
\begin{titlepage}
\newcommand\myhlinewotikz{\noindent\rule[.35em]{0.95\linewidth}{.4pt}\par}
\center

\begin{flushleft} \large
\textsc{\Large{Univerzitet u Tuzli\\Fakultet elektrotehnike}}
\end{flushleft}

\vspace{\stretch{1}}
%\rule{\textwidth}{1pt}
\myhlinewotikz
\textsc{\LARGE{Glavni Naslov}}\par
\vspace{3mm}
\textsc{\LARGE{Podnaslov}}\par
\myhlinewotikz

\vspace{\stretch{2.5}}
\noindent
\begin{minipage}{0.5\textwidth}
\begin{flushleft} \large
\emph{Autor:}\\
Asmir Abdulahović
\end{flushleft}
\end{minipage}%
\begin{minipage}{0.5\textwidth}
\begin{flushright} \large
\emph{Prof:} \\
dr.sci. Profime Profprez
\end{flushright}
\end{minipage}

\end{titlepage}


\end{document}

%% Include figure
% \begin{figure}[H]
% 	\centering
% 	% \includegraphics[width=\textwidth]{druga.png}
% 	%\resizebox{\textwidth}{!}{\input{zad2.tikz}}
% 	\caption*{Slika 2: Poklapanje sa originalnom funkcijom}
% \end{figure}
